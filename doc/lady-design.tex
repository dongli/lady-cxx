\documentclass[slidestop,compress,mathserif]{beamer}
\usetheme{Antibes}
\usecolortheme{spruce}
\definecolor{main-color}{HTML}{005128}
\setbeamertemplate{itemize item}{\color{main-color}$\blacktriangleright$}

\title{Design of LADY v0.1.0}
\author{Li Dong}

\begin{document}

\begin{frame}
  \titlepage
\end{frame}

\section{Rationale}
\label{sec:Rationale}

\begin{frame}

\end{frame}

\section{Discretization}
\label{sec:Discretization}

\begin{frame}
  Mass density
  \begin{equation}
    \rho(\mathbf{x}) = \sum_{i = 1}^{N} m_i \psi_i(\mathbf{x})
  \end{equation}
  Mass flux
  \begin{equation}
    \rho(\mathbf{x}) \mathbf{v}(\mathbf{x}) = \sum_{i = 1}^{N} m_i \psi_i(\mathbf{x}) \left[ \mathbf{v}_i + d\mathbf{H}_i \mathbf{H}_i^{-1}(\mathbf{x} - \mathbf{x}_i) \right]
  \end{equation}
  \begin{exampleblock}{Shape function and body coordinate}
    \begin{equation*}
      \psi_i(\mathbf{x}) = \frac{\psi(\mathbf{y})}{\det{\mathbf{H}}_i}\text{, where }\mathbf{y} = \mathbf{H}_i^{-1}(\mathbf{x} - \mathbf{x}_i))
    \end{equation*}
  \end{exampleblock}
\end{frame}

\section{Barotropic model}
\label{sec:Barotropic model}

\begin{frame}
  Notes on the model:
  \begin{itemize}
    \item Density is a function of pressure only and vice versa

    \item Density is constant, $\rho$ is replaced by $h$ (height)
    \item Incompressible in 3D, but compressible in 2D view (height changes)
  \end{itemize}
\end{frame}

\begin{frame}
  Geopotential energy:
  \begin{equation}
    E_p = \frac{1}{2} \int g h^2(\mathbf{x}) d\mathbf{x}
    \label{eqn:geopotential-energy}
  \end{equation}
  \begin{exampleblock}{Height is diagnosed from fluid continuity.}
    \begin{equation*}
      h(\mathbf{x}) = \sum_{i = 1}^{N} m_i \psi_i(\mathbf{x}).
    \end{equation*}
  \end{exampleblock}
\end{frame}

\begin{frame}
  Momentum equations:
  \begin{align}
    \frac{d \mathbf{v}_i}{d t} & = {\mathbf{F}_p^{(g)}}_i \\
    \frac{d \dot{\mathbf{H}}_i}{d t} & = {\mathbf{M}_p^{(g)}}_i
  \end{align}
  where ${\mathbf{F}_p^{(g)}}$ and ${\mathbf{M}_p^{(g)}}$ are pressure gradient forces induced by gravity and fluid continuity.
  \begin{align}
    {\mathbf{F}_p^{(g)}}_i & = - \frac{1}{m_i} \frac{\partial E_p}{\partial \mathbf{x}_i}, \\
    {\mathbf{M}_p^{(g)}}_i & = - \frac{1}{m_i J} \frac{\partial E_p}{\partial \mathbf{H}_i}.
  \end{align}
\end{frame}

\begin{frame}
  Using quadrature points to approximate spatial integration in Eq. (\ref{eqn:geopotential-energy}):
  \begin{equation}
    E_p = \frac{1}{2} \int g h^2(\mathbf{x}) d\mathbf{x} \approx \frac{1}{2} \sum_{j = 1}^{N} m_j \sum_{k = 1}^{n} w_k g h(\mathbf{x}_j^k).
  \end{equation}
  \begin{exampleblock}{Replace continuous integration by discrete one}
    \begin{equation*}
      \int{f(\mathbf{y}) \psi(\mathbf{y}) d\mathbf{y}} \approx \sum_{k = 1}^{n} w_k f(\mathbf{y}_k),
    \end{equation*}
    where $w_k$ is the quadrature point weight.
  \end{exampleblock}
\end{frame}

\begin{frame}
  \begin{equation*}
    \frac{\partial E_p}{\partial \mathbf{x}_i} = {\frac{\partial E_p}{\partial \mathbf{x}_i}}^{(1)} + {\frac{\partial E_p}{\partial \mathbf{x}_i}}^{(2)}, \quad
    \frac{\partial E_p}{\partial \mathbf{H}_i} = {\frac{\partial E_p}{\partial \mathbf{H}_i}}^{(1)} + {\frac{\partial E_p}{\partial \mathbf{H}_i}}^{(2)}
  \end{equation*}
  \begin{align}
    {\frac{\partial E_p}{\partial \mathbf{x}_i}}^{(1)} & = \frac{1}{2} \sum_{j = 1}^{N} m_j \sum_{k = 1}^{n} w_k g m_i \left.\frac{\partial \psi_i}{\partial \mathbf{x}_i}\right|_{\mathbf{x}_j^k} \\
    {\frac{\partial E_p}{\partial \mathbf{H}_i}}^{(1)} & = \frac{1}{2} \sum_{j = 1}^{N} m_j \sum_{k = 1}^{n} w_k g m_i \left.\frac{\partial \psi_i}{\partial \mathbf{H}_i}\right|_{\mathbf{x}_j^k} \\
    {\frac{\partial E_p}{\partial \mathbf{x}_i}}^{(2)} & = \frac{1}{2} m_i \sum_{k = 1}^{n} w_k g m_i \left.\frac{\partial \psi_i}{\partial \mathbf{x}}\right|_{\mathbf{x}_j^k} \\
    {\frac{\partial E_p}{\partial \mathbf{H}_i}}^{(2)} & = \frac{1}{2} m_i \sum_{k = 1}^{n} w_k g m_i \left.\frac{\partial \psi_i}{\partial \mathbf{x}}\right|_{\mathbf{x}_j^k} \mathbf{y}_k^T
  \end{align}
\end{frame}

\section{Baroclinic model}
\label{sec:Baroclinic model}

\begin{frame}

\end{frame}

\section{Temporal integrator}
\label{sec:Temporal integrator}

\begin{frame}
  Leap frog scheme:
  \begin{center}
    \includegraphics[width=0.9\textwidth]{leap-frog-diagram.pdf}
  \end{center}
\end{frame}

\end{document}
